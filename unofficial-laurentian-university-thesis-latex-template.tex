\documentclass[openany, 12pt]{book}

% PACKAGE IMPORTS
\usepackage{import}
\usepackage{titling}
\usepackage[letterpaper, margin=1in]{geometry}
\usepackage{setspace}
\usepackage{nomencl}
\usepackage{newfloat}
\usepackage{tocloft}
\usepackage{fancyhdr}
\usepackage{booktabs}
\usepackage{graphicx}
\usepackage{titlesec}
\usepackage{natbib}
\usepackage{amsmath}
\usepackage{caption}

% DOCUMENT SETTINGS AND MACROS

% Thesis properties
\newcommand{\givenname}{Stephen}
\newcommand{\surname}{Young}
\newcommand{\program}{Program}
\newcommand{\programdegree}{Degree}
\newcommand{\degreeAbbr}{Abbr.}
\newcommand{\dateofdefence}{\today}
\title{Unofficial Laurentian University thesis \LaTeX{} template}
\author{\givenname\ \surname}

% Paragraph settings
\setlength{\parindent}{0pt}
\setlength{\parskip}{12pt}
\setlength{\headheight}{14.5pt}
\addtolength{\topmargin}{-2.5pt}

% Chapter display style
\newcommand{\titlefont}{\bfseries\Large}
\newcommand{\frontchap}[1]{
    \clearpage
    \begin{center} \titlefont #1 \end{center}
    \addcontentsline{toc}{chapter}{#1}
}
\titleformat{\chapter}[display]{\titlefont}{\centering\chaptertitlename{} \thechapter}{0ex}{}[]
\titlespacing*{\chapter}{0pt}{0pt}{0pt}

% Page style settings
\fancypagestyle{frontmatter}{
    \fancyhf{}
    \fancyfoot[C]{\thepage}
    \renewcommand{\headrulewidth}{0pt}
    \renewcommand{\footrulewidth}{0pt}
}
\fancypagestyle{plain}{
    \fancyhf[R]{\thepage}
    \fancyfoot{}
    \renewcommand{\headrulewidth}{0pt}
    \renewcommand{\footrulewidth}{0pt}
}
\tocloftpagestyle{frontmatter}

% Table of contents settings
\renewcommand{\contentsname}{Table of Contents}
\renewcommand{\cfttoctitlefont}{\hfill\titlefont}
\renewcommand{\cftaftertoctitle}{\hfill}
\renewcommand{\cftchapfont}{\normalsize}
\renewcommand{\cftchappagefont}{\normalsize}
\renewcommand{\cftchappresnum}{\chaptername{}~}
\renewcommand{\cftchapleader}{\cftdotfill{\cftdotsep}}
\setlength{\cftaftertoctitleskip}{1.5em}
\setlength{\cftchapnumwidth}{2.5cm}
\newlength{\entrysep}
\setlength{\entrysep}{6pt}
\setlength{\cftbeforechapskip}{\entrysep}
\setlength{\cftbeforesecskip}{\entrysep}
\setlength{\cftbeforesubsecskip}{\entrysep}

% List of tables settings
\newcommand*{\loftvspace}[1]{
    \renewcommand*{\addvspace}[1]{\vspace{#1}}
}
\renewcommand{\cftlottitlefont}{\hfill\titlefont}
\renewcommand{\cftafterlottitle}{\hfill}
\renewcommand{\cfttabpresnum}{Table~}
\renewcommand{\cfttabaftersnum}{:}
\setlength{\cftafterlottitleskip}{1.5em}
\setlength{\cfttabnumwidth}{2.0cm}
\setlength{\cftbeforetabskip}{\entrysep}
\addtocontents{lot}{\protect\loftvspace{0pt}}

% List of figures settings
\renewcommand{\cftloftitlefont}{\hfill\titlefont}
\renewcommand{\cftafterloftitle}{\hfill}
\renewcommand{\cftfigpresnum}{Figure~}
\renewcommand{\cftfigaftersnum}{:}
\setlength{\cftafterloftitleskip}{1.5em}
\setlength{\cftfignumwidth}{2.2cm}
\setlength{\cftbeforefigskip}{\entrysep}
\addtocontents{lof}{\protect\loftvspace{0pt}}

% Plates environment and list of plates settings
\newlistof{plate}{lop}{\listplatename}
\DeclareFloatingEnvironment[
    fileext=lop,
    name=Plate,
    placement=tbhp,
    within=chapter
]{plate}
\renewcommand{\listplatename}{List of Plates}
\renewcommand{\cftloptitlefont}{\hfill\titlefont}
\renewcommand{\cftafterloptitle}{\hfill}
\renewcommand{\cftplatepresnum}{Plate~}
\renewcommand{\cftplateaftersnum}{:}
\setlength{\cftafterloptitleskip}{1.5em}
\setlength{\cftplatenumwidth}{2.0cm}
\setlength{\cftbeforeplateskip}{\entrysep}
\addtocontents{lop}{\protect\loftvspace{0pt}}

% List of appendices settings
\newlistof{appendices}{loa}{\listappendixname}
\newcommand{\listappendixname}{List of Appendices}
\renewcommand{\cftloatitlefont}{\hfill\titlefont}
\renewcommand{\cftafterloatitle}{\hfill}
\renewcommand{\cftappendicespresnum}{\appendixname}
\setlength{\cftafterloatitleskip}{1.5em}
\newcommand{\startappendices}{
    \clearpage
    \appendix
    \frontchap{Appendices}
}
\newcommand{\newappendix}[1]{
    \refstepcounter{appendices}
    \refstepcounter{chapter}
    \section*{\appendixname{} \Alph{appendices}: #1}
    \addcontentsline{toc}{chapter}{\appendixname{} \Alph{appendices}: #1}
    \addcontentsline{loa}{appendices}{\appendixname{} \Alph{appendices}: #1}
}

% Nomenclature settings
\renewcommand{\nomname}{\hfill Nomenclature \hfill}

\renewcommand{\nomgroup}[1]{%
% \ifthenelse{\equal{#1}{a}}{\item[\bfseries{Roman letters}]}{% Uncomment these lines for an explicit "Roman letters" section for the nomenclature
% \ifthenelse{\equal{#1}{A}}{\item[\bfseries{Roman letters}]}{% Matching closing braces will need to be added also.
\ifthenelse{\equal{#1}{g}}{\item[\bfseries{Greek letters}]}{%
\ifthenelse{\equal{#1}{G}}{\item[\bfseries{Greek letters}]}{%
\ifthenelse{\equal{#1}{x}}{\item[\bfseries{Superscripts}]}{%
\ifthenelse{\equal{#1}{X}}{\item[\bfseries{Superscripts}]}{%
\ifthenelse{\equal{#1}{Z}}{\item[\bfseries{Subscripts}]}{%
\ifthenelse{\equal{#1}{z}}{\item[\bfseries{Subscripts}]}{}}}}}}}
\makenomenclature

% List of references settings
\renewcommand{\bibname}{\hfill References\hfill}
\renewcommand{\bibfont}{\raggedright\singlespacing}
\bibliographystyle{dcu}
\setcitestyle{aysep={},citesep={,}}

\captionsetup{justification=centering,font=bf}

\graphicspath{{./figures/}}

% THESIS DOCUMENT
\begin{document}

\import{./contents/}{frontmatter.tex}

\chapter{Chapter titles use the \texttt{chapter} command}

\section{First level subheading uses \texttt{section} command}

A sample table is shown in table \ref{tab:sample-table}.
Use the \texttt{booktabs} and \texttt{caption} packages for better table and caption formatting.

\begin{table}
    \centering
    \caption{Sample table which will automatically update in List of Tables}
    \label{tab:sample-table}
    \import{./tables/}{tab-sample-table.tex}
\end{table}

\subsection{Second level subheading uses \texttt{subsection} command}

\chapter{Chapter titles use the \texttt{chapter} command}

\section{First level subheading uses \texttt{section} command}

\begin{figure}
    \centering
    \includegraphics[width=3.5cm]{laurentian-university-coa.png}
    \caption{Sample figure. Use the \texttt{figure} environment for figures to automatically populate in the list of figures}
\end{figure}

\subsection{Second level subheading uses \texttt{subsection} command}

\chapter{Chapter titles use the \texttt{chapter} command}

\section{First level subheading uses \texttt{section} command}

The custom \texttt{plate} environment is added to this document. If it is not required, remove the list of plates declaration in the front matter.

\begin{plate}
    \centering
    \includegraphics[width=3.5cm]{laurentian-university-coa.png}
    \caption{Sample plate. Use the \texttt{plate} environment for plates to automatically populate in the list of plates}
\end{plate}

\subsection{Second level subheading uses \texttt{subsection} command}

\chapter{Chapter titles use the \texttt{chapter} command}

\section{First level subheading uses \texttt{section} command}

Use the \texttt{equation} environment for numbered equations.
Use \texttt{nomenclature} commands to add the symbols to nomenclature at their first use.

\begin{equation}
    A_w = \frac{\pi l}{2}\left(d_1 + d_2\right)
\end{equation}

\nomenclature[aA]{\(A\)}{Area}
\nomenclature[ad]{\(d\)}{Diameter}
\nomenclature[al]{\(l\)}{Segment slanted wall length}
\nomenclature[gp]{\(\pi\)}{Dimensionless mathematical constant}
\nomenclature[zw]{\(w\)}{Segment wall}
\nomenclature[z1]{\(1\)}{Segment inlet}
\nomenclature[z2]{\(2\)}{Segment outlet}

\begin{equation}
    H^{cp} = H^\ominus \exp\left[K_H\left(\frac{1}{T}-\frac{1}{T^\ominus}\right)\right]
\end{equation}

\nomenclature[aH]{\(H\)}{Henry's constant}
\nomenclature[aK]{\(K_H\)}{Henry's constant temperature coefficient}
\nomenclature[aT]{\(T\)}{Temperature}
\nomenclature[xo]{\(\ominus\)}{Value at standard conditions}
\nomenclature[xc]{\(cp\)}{Defined by concentration and pressure}

The \texttt{amsmath} package provides other useful math environments.
For example, the \texttt{gather} environment enables the grouping of equations with each equation numbered.
\begin{gather}
    \dot{m}_{l,1}e_{l,1} + \dot{m}_{g,1}e_{g,1} = \dot{m}_{l,2}e_{l,2} + \dot{m}_{g,2}e_{g,2} \label{eq:energy-conservation} \\
    e_{l,k} = \frac{P_k}{\rho_{l,k}} + \frac{U^2_{l,k}}{2} + gz_k + u_{l,k} \label{eq:liquid-energy} \\
    e_{g,k} = \frac{P_k}{\rho_{g,k}} + \frac{\left(U_{l,k}-U_{s,k}\right)^2}{2} + gz_k + u_{g,k} \label{eq:gas-energy}
\end{gather}

\subsection{Second level subheading uses \texttt{subsection} command}

\chapter{Chapter titles use the \texttt{chapter} command}

\section{First level subheading uses \texttt{section} command}

The \texttt{natbib} package is used for formatting citations and references/bibliography section.
Collect your references with the software of your choice (e.g., Zotero) and export your references to \texttt{BibTeX} format and add them to the \texttt{.bib} file.
Harvard style referencing has been implemented in this template.
See the \texttt{natbib} documentation if a different style is required.

In-text citations are done with the \texttt{citet} command while parenthetical citations are done with the \texttt{citep} command.
Example of an in-text citation: \citet{sander2015}.
Example of an parenthetical citation: \citep{sander2015}.
Multiple citation (in-text): \citet{sander2015,akita1974}.
Multiple citation (parenthetical): \citep{sander2015,akita1974}.
Citation with repeating author: \citep{millar2016,millar2018}.
Citation with repeating author: \citet{millar2016,millar2018}.

\subsection{Second level subheading uses \texttt{subsection} command}

% References section
\clearpage
\addcontentsline{toc}{chapter}{References}
\bibliography{unofficial-laurentian-university-thesis-latex-template}

% Appendices
\startappendices
\newappendix{Sample Appendix}

Use the \texttt{startappendices} macro to start the appendices section.
Add appendices with the \texttt{newappendix} macro and entries in the list of appendices and table of contents will automatically generated.

\clearpage % Use \clearpage to insert page break between appendices
\newappendix{Another Appendix}

\end{document}
